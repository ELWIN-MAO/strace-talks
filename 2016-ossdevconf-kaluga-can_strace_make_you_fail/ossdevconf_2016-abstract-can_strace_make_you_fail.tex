% Copyright (C) 2012-2016  Dmitry V. Levin <ldv@altlinux.org>
% Permission is granted to copy, distribute and/or modify this document
% under the terms of the GNU Free Documentation License, Version 1.2
% or any later version published by the Free Software Foundation;
% with no Invariant Sections, no Front-Cover Texts, and no Back-Cover Texts.

\author{Dmitry V. Levin}
\city{Moscow}
\affiliation{BaseALT}
\projecttitle{strace}
\projecturl{\url{https://strace.io}}
\title{Can strace make you fail?}

\maketitle

\begin{abstract}
strace is a diagnostic, debugging and instructional userspace utility
for Linux.  It is used to monitor interactions between processes and the
Linux kernel, which include system calls, signal deliveries, and changes
of process state.  This year strace has been extended to tamper with
tracees using controllable syscall fault injection.
\end{abstract}

\section{Введение}

strace как инструмент мониторинга взаимодействия пользовательских
процессов с ядром существует уже почти 25 лет и широко применяется для
диагностики, отладки и изучения поведения ПО.  Многочисленные параметры
управления фильтрацией дают возможность пользователю strace легко и
гибко настраивать отображение системных вызовов и сигналов.  С каждым
выпуском strace таких возможностей становится больше, а точность
отображения -- выше.

В этом году Nahim El Atmani, студент лаборатории LSE в École pour
l'informatique et les techniques avancées (EPITA), в рамках одного из
проектов GSoC 2016\cite{gsoc} реализовал прототип новой
функции\cite{nahim} в strace, которая принципиально отличается от всех
предыдущих и открывает новые направления применения strace.

\section{fault injection как метод тестирования}

Как тестировать обработку программой нетривиально воспроизводимых
ситуаций вообще, и особенно ошибок, которые не происходят во время
тестирования?  Например, можно искусственным образом создавать у
программы представление о том, что тестируемое условие произошло.  Но
как это сделать, не внося изменений в тестируемую программу?  Можно
попробовать изменять поведение системных вызовов, например, с помощью
strace!

\section{strace и системные вызовы}

strace отслеживает системные вызовы, выполняемые подопечными процессами,
с помощью системных вызовов ptrace и wait4.  ptrace-запросами
PTRACE\_SYSCALL strace поручает ядру останавливать трассируемые процессы
при выполнении ими системных вызовов.  Эти процессы на входе в любой
системный вызов переводятся ядром в состояние syscall-enter-stop, а на
выходе из любого системного вызова -- в состояние syscall-exit-stop.  Об
изменении состояния процессов ядро оповещает strace, ожидающий событий в
wait4.

strace всякий раз, обнаруживая трассируемый процесс в состоянии
syscall-enter-stop, выясняет номер и параметры системного вызова, после
чего применяет фильтры, которые на основе этой информации формируют
решение о том, отображать ли этот системный вызов.  Если системный вызов
подлежит отображению, выполняется соответствующий парсер, который
согласно настройкам и семантике этого системного вызова отображает его
надлежащим образом.  Обработав состояние syscall-enter-stop, strace дает
ядру запрос PTRACE\_SYSCALL, после которого ядро продолжает выполнять
приостановленный системный вызов до состояния syscall-exit-stop.
Обнаружив трассируемый процесс в этом состоянии, strace при
необходимости выясняет и отображает код возврата и, возможно, другие
результаты работы системного вызова.  Обработав состояние
syscall-exit-stop, strace снова дает ядру запрос PTRACE\_SYSCALL, после
которого трассируемый процесс продолжает работу до следующего системного
вызова.

\section{fault injection системных вызовов}

ptrace API позволяет не только считывать номер, параметры, и код
возврата системного вызова, но и менять их.  Например, путем замены
номера системного вызова в состоянии syscall-enter-stop на $-1$
происходит замена системного вызова на заведомо несуществующий, а замена
кода возврата системного вызова в состоянии syscall-exit-stop приводит к
установке произвольного кода возврата.

Реализованный в strace интерфейс позволяет осуществлять fault injection
произвольного множества системных вызовов, как всех подряд, так и
выборочно, например, только $N$-й вызов системного вызова, каждый $N$-й
вызов, и т.п.  В сочетании с традиционным фильтром по имени файла можно
осуществлять fault injection системных вызовов, прямо или косвенно
оперирующих определенными файлами.

\section{Первые находки}

Экспериментальные запуски strace в режиме syscall fault injection сразу
выявили проблемы с обработкой ошибок в разных программах.  Например,
python3 не обрабатывает ошибку доступа к устройству /dev/urandom:
{\tt\tiny
\begin{verbatim}
$ strace -P /dev/urandom -e fault=open:1:ENOENT python3
open("/dev/urandom", O_RDONLY|O_CLOEXEC) = -1 ENOENT (No such file or directory) (INJECTED)
Fatal Python error: Failed to open /dev/urandom
--- SIGSEGV {si_signo=SIGSEGV, si_code=SEGV_MAPERR, si_addr=0x50} ---
+++ killed by SIGSEGV +++
Segmentation fault
\end{verbatim}
}

Еще один пример показывает, что динамический загрузчик не всегда
проверяет код возврата системного вызова mprotect:
{\tt\tiny
\begin{verbatim}
$ strace -e mprotect -e fault=mprotect:1:EPERM pwd > /dev/null
mprotect(0x7fabcd00f000, 2097152, PROT_NONE) = -1 EPERM (Operation not permitted) (INJECTED)
mprotect(0x7fabcd20f000, 16384, PROT_READ) = 0
mprotect(0x606000, 4096, PROT_READ)     = 0
mprotect(0x7fabcd441000, 4096, PROT_READ) = 0
+++ exited with 0 +++
$ strace -e mprotect -e fault=mprotect:2:EPERM pwd > /dev/null
mprotect(0x7fabcd00f000, 2097152, PROT_NONE) = 0
mprotect(0x7fabcd20f000, 16384, PROT_READ) = -1 EPERM (Operation not permitted) (INJECTED)
pwd: error while loading shared libraries: /lib64/libc.so.6: cannot apply additional memory
protection after relocation: Operation not permitted
+++ exited with 127 +++
\end{verbatim}
}

\section{Что дальше?}

Продолжением и расширением функции syscall fault injection могла бы
стать функция syscall success injection.  Тот же прием, который был
использован для реализации fault injection, годится и для создания
видимости успешного завершения системного вызова.  Основная сложность в
реализации success injection -- сохранение семантики системных вызовов,
результатами работы которых является не только код возврата, но и,
например, запись определенных адресов оперативной памяти.

Область применения strace syscall fault injection -- не только
расширение тестового покрытия и поиск ошибок.  Например, с помощью этой
функции непривилегированный пользователь может легко организовать
временное блокирование отдельных системных вызовов и обращений к
отдельным файлам у определенных процессов, не прибегая к написанию
seccomp filters.


\begin{thebibliography}{9}

\bibitem{gsoc} Google Summer of Code: strace. \\
 \url{https://summerofcode.withgoogle.com/organizations/5106770607341568/}

\bibitem{nahim} strace syscall fault injection. \\
 \url{https://brokenpi.pe/tools/strace-fault-injection}

\end{thebibliography}
