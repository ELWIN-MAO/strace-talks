% Copyright (C) 2012  Dmitry V. Levin <ldv@altlinux.org>
% Permission is granted to copy, distribute and/or modify this document
% under the terms of the GNU Free Documentation License, Version 1.2
% or any later version published by the Free Software Foundation;
% with no Invariant Sections, no Front-Cover Texts, and no Back-Cover Texts.

\author{Дмитрий Левин}
\city{Москва}
\affiliation{ALT Linux}
\projecttitle{strace,}
\projecturl{\url{http://sourceforge.net/projects/strace/}}
% \date{12 июля 2012~г.}
\title{strace глазами апстрима}

\maketitle

\begin{abstract}
Напоминается история проекта,
рассматриваются основные возможности \emph{strace}
и нововведения в недавно выпущенных версиях,
приводятся примеры использования \emph{strace},
дается обзор \emph{ptrace API} и реализации \emph{strace},
рассматриваются перспективы дальнейшего развития.
\end{abstract}

\section*{История проекта}
\begin{itemize}
\item 1991-1992: Paul Kranenburg;
\item 1993: Branko Lankester;
\item 1993-1996: Richard Sladkey;
\item 1996-1998: смутное время;
\item 1998-2002: Wichert Akkerman;
\item 2002-2009: Roland McGrath;
\item с 2009.
\end{itemize}

\section*{Основные возможности и примеры использования}
\begin{itemize}
\item Отслеживание порождаемых процессов: -f, -ff.
\item Мониторинг процессов, существующих на момент запуска: -p.
\item Сбор статистики: -c, -C, -S.
\item Дополняющие описатели: -i, -r, -t, -tt, -ttt, -T.
\item Формат вывода строк: -s, -x, -xx.
\item Выбор множества отслеживаемых системных вызовов: -e trace=set.
\item Глубина детализации множества системных вызовов:
-v, -e abbrev=set, -e verbose=set, -e raw=set.
\item Выбор множества отслеживаемых сигналов: -e signal=set.
\item Мониторинг операций чтения/записи: -e read=set, -e write=set.
\item Конвейеризация: -o.
\item buildreq.
\end{itemize}

\section*{Нововведения с 2009 года}
\begin{itemize}
\item 4.5.19: прозрачность, новые архитектуры, новые системные вызовы, улучшенные парсеры.
\item 4.5.20: новый ключ -C, новые архитектуры, новые системные вызовы, улучшенные парсеры.
\item 4.6: переход на новый Linux ptrace API для реализации отслеживания процессов,
новые архитектуры, новые системные вызовы, улучшенные парсеры.
\item 4.7: новые ключи (-y, -P, -I), новые архитектуры, новые системные вызовы, улучшенные парсеры.
\end{itemize}

\section*{Обзор реализации}
\begin{itemize}
\item Linux PTRACE API.
\item Отслеживание процессов.
\item Парсеры системных вызовов.
\item Реализация multiarch.
\end{itemize}

\section*{Перспективы дальнейшего развития}
\begin{itemize}
\item Интерфейс PTRACE\_SEIZE.
\item Все более и более детализованные парсеры.
\item Надежный multiarch.
\end{itemize}
